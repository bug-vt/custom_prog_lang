%%
%% This is file `sample-manuscript.tex',
%% generated with the docstrip utility.
%%
%% The original source files were:
%%
%% samples.dtx  (with options: `manuscript')
%% 
%% IMPORTANT NOTICE:
%% 
%% For the copyright see the source file.
%% 
%% Any modified versions of this file must be renamed
%% with new filenames distinct from sample-manuscript.tex.
%% 
%% For distribution of the original source see the terms
%% for copying and modification in the file samples.dtx.
%% 
%% This generated file may be distributed as long as the
%% original source files, as listed above, are part of the
%% same distribution. (The sources need not necessarily be
%% in the same archive or directory.)
%%
%% Commands for TeXCount
%TC:macro \cite [option:text,text]
%TC:macro \citep [option:text,text]
%TC:macro \citet [option:text,text]
%TC:envir table 0 1
%TC:envir table* 0 1
%TC:envir tabular [ignore] word
%TC:envir displaymath 0 word
%TC:envir math 0 word
%TC:envir comment 0 0
%%
%%
%% The first command in your LaTeX source must be the \documentclass command.
%%%% Small single column format, used for CIE, CSUR, DTRAP, JACM, JDIQ, JEA, JERIC, JETC, PACMCGIT, TAAS, TACCESS, TACO, TALG, TALLIP (formerly TALIP), TCPS, TDSCI, TEAC, TECS, TELO, THRI, TIIS, TIOT, TISSEC, TIST, TKDD, TMIS, TOCE, TOCHI, TOCL, TOCS, TOCT, TODAES, TODS, TOIS, TOIT, TOMACS, TOMM (formerly TOMCCAP), TOMPECS, TOMS, TOPC, TOPLAS, TOPS, TOS, TOSEM, TOSN, TQC, TRETS, TSAS, TSC, TSLP, TWEB.
% \documentclass[acmsmall]{acmart}

%%%% Large single column format, used for IMWUT, JOCCH, PACMPL, POMACS, TAP, PACMHCI
% \documentclass[acmlarge,screen]{acmart}

%%%% Large double column format, used for TOG
% \documentclass[acmtog, authorversion]{acmart}

%%%% Generic manuscript mode, required for submission
%%%% and peer review
\documentclass[manuscript,screen,nonacm]{acmart}
%% Fonts used in the template cannot be substituted; margin 
%% adjustments are not allowed.
%%
%% \BibTeX command to typeset BibTeX logo in the docs
\AtBeginDocument{%
  \providecommand\BibTeX{{%
    \normalfont B\kern-0.5em{\scshape i\kern-0.25em b}\kern-0.8em\TeX}}}

%% Rights management information.  This information is sent to you
%% when you complete the rights form.  These commands have SAMPLE
%% values in them; it is your responsibility as an author to replace
%% the commands and values with those provided to you when you
%% complete the rights form.

%% These commands are for a PROCEEDINGS abstract or paper.

%
%  Uncomment \acmBooktitle if th title of the proceedings is different
%  from ``Proceedings of ...''!
%


%%
%% Submission ID.
%% Use this when submitting an article to a sponsored event. You'll
%% receive a unique submission ID from the organizers
%% of the event, and this ID should be used as the parameter to this command.
%%\acmSubmissionID{123-A56-BU3}

%%
%% For managing citations, it is recommended to use bibliography
%% files in BibTeX format.
%%
%% You can then either use BibTeX with the ACM-Reference-Format style,
%% or BibLaTeX with the acmnumeric or acmauthoryear sytles, that include
%% support for advanced citation of software artefact from the
%% biblatex-software package, also separately available on CTAN.
%%
%% Look at the sample-*-biblatex.tex files for templates showcasing
%% the biblatex styles.
%%

%%
%% The majority of ACM publications use numbered citations and
%% references.  The command \citestyle{authoryear} switches to the
%% "author year" style.
%%
%% If you are preparing content for an event
%% sponsored by ACM SIGGRAPH, you must use the "author year" style of
%% citations and references.
%% Uncommenting
%% the next command will enable that style.
%%\citestyle{acmauthoryear}

%%
%% end of the preamble, start of the body of the document source.
\begin{document}

%%
%% The "title" command has an optional parameter,
%% allowing the author to define a "short title" to be used in page headers.
\title{CS 4974 Independent Study: Scripting language design and implementation}

%%
%% The "author" command and its associated commands are used to define
%% the authors and their affiliations.
%% Of note is the shared affiliation of the first two authors, and the
%% "authornote" and "authornotemark" commands
%% used to denote shared contribution to the research.
\author{Bug Lee}
\email{bug19@vt.edu}
\affiliation{%
  \institution{Virginia Tech}
  \city{Blacksburg}
  \state{Virginia}
  \country{USA}
  \postcode{24061}
}
\makeatletter
\let\@authorsaddresses\@empty
\makeatother

%%
%% By default, the full list of authors will be used in the page
%% headers. Often, this list is too long, and will overlap
%% other information printed in the page headers. This command allows
%% the author to define a more concise list
%% of authors' names for this purpose.
\renewcommand{\shortauthors}{Lee.}

%%
%% The abstract is a short summary of the work to be presented in the
%% article.
\begin{abstract}

\end{abstract}

%%
%% The code below is generated by the tool at http://dl.acm.org/ccs.cfm.
%% Please copy and paste the code instead of the example below.
%%
\begin{CCSXML}
<ccs2012>
   <concept>
       <concept_id>10011007.10011006.10011041</concept_id>
       <concept_desc>Software and its engineering~Compilers</concept_desc>
       <concept_significance>500</concept_significance>
       </concept>
 </ccs2012>
\end{CCSXML}

\ccsdesc[500]{Software and its engineering~Compilers}

%%
%% Keywords. The author(s) should pick words that accurately describe
%% the work being presented. Separate the keywords with commas.
\keywords{Compiler, Assembler, Virtual Machine}

%% A "teaser" image appears between the author and affiliation
%% information and the body of the document, and typically spans the
%% page.

%%
%% This command processes the author and affiliation and title
%% information and builds the first part of the formatted document.
\maketitle
\cite{Nystrom}
\section{Introduction}
Over the years, the complexity of compiler design has immensely increased. At the hardware level, there are more diverse and sophisticated instruction sets to consider. At the software level, complex optimization schemes have been introduced for maximal performance, in addition to different language paradigms. As a result, the complexity of the modern compiler design overshadows the fundamentals behind the language construction as well as making the steps between programming language and execution more mysterious. For this reason, this project focused on two aspects: (1) building a simplified version of each component used in programming language construction, and (2) understanding how each component works together to execute a custom programming language. This report describes the semester-long project of building a C-like (C subset) programming language from scratch. 

\section{Project Overview}
The project was broken down into 4 parts, ordered from lower level to higher level: Virtual Machine, Assembler, Compiler, and Fuzzer. Each level depends on the level right below it, except for Virtual Machine which is a standalone software.

\subsection{Virtual Machine}
The Virtual Machine mimics the generic single-cycle hardware processor. The following describes the similarity:
\begin{itemize}
    \item Load instructions from the executable to the instruction cache. 
    \item Load and store a value into runtime stack and keep track of stack frame using stack pointer and frame pointer.
    \item Use the return register to access the return value from a function.
    \item Use the instruction pointer to read and execute the next instruction.
    \item Execute one instruction in a single cycle.
\end{itemize}
However, since the Virtual Machine was implemented using C++ instead of transistors, it was flexible for more functionalities that are not available in generic hardware:
\begin{itemize}
    \item Include the function cache that load all the function information from the executable.
    \item Perform type resolution/casting/coercion during runtime.
\end{itemize}
The Virtual Machine support 36 instructions (see appendix). The instruction set for Virtual Machine was designed to follow the Complex Instruction Set Computing (CISC) methodology. This is to make the runtime environment faster: doing as much work as possible in C++ instead of leaving implementation to a slower custom language.

Overall, the role of the Virtual Machine is to set the first point of simplification. Even with the simpler instruction set, however, writing binary executables by hand would be a painful task. Which lead to the implementation of the assembler.

Note that the Virtual Machine design was adapted from the Varanese's XVM\cite{Varanese}. However, unlike XVM, it was simplified to single-threaded and stand-alone software. Also, new instructions were added to support pass-by-reference and print functionality. 

\subsection{Assembler}
Like other assemblers for specific hardware processors, the main functionality is to allow users to use mnemonics and numeric operands, which then get translated to binary executables. However, the custom assembler supports more advanced features and takes advantage of flexibility from the Virtual Machine:
\begin{itemize}
    \item Differentiate function and label. Use func directive and curly braces to define function. Automatically add ret instruction at the end of the function after assembling.
    \item Use Var and Param directives to define variables or arrays and automatically reserve space inside the stack.
    \item Support basic scope resolution. 
    \item Direct support for string type.
    \item Allow instruction to accept a string, variable, and array (absolute or variable index). Note that type checking is done by a Virtual Machine during the runtime.
\end{itemize}
However, the limitation of expressivity and tedium of controlling the runtime stack still make assembly language difficult to work with and error-prone. The next subsection, compiler, improve on this issue.

Note that the Assembler design was adapted from the Varanese's XASM\cite{Varanese}. However, there are two major differences compared to XASM. First, the lexer for the custom assembler was implemented using regular expression and state machine whereas XASM used a brute force approach. Secondly, although both the custom assembler and XASM implement a top-down parser, the parser for the custom assembler resembles more closely to recursive descent than XASM. 

\subsection{Compiler}

\section{Virtual Machine}

\subsection{Runtime Stack}
The Virtual Machine simulates the runtime stack using the linked list. Each stack position can hold any value type (integer, float, stack index, table index, and register code) and increment by 1, resembling word addressable memory. All values were designed to fit inside a single stack position, simplifying stack access/management. String processing instructions are available when a character must be accessed/modified inside the string

\subsection{Execution Cycle}
The execution cycle begins by first loading the entry point from the executable. Once load the entry point, the Virtual Machine takes the following steps for each cycle, which resemble the execution cycle of the hardware processor:
\begin{itemize}
    \item Fetch instruction \\
	Read the instruction that the instruction pointer is pointing at.
    \item Decode opcode \\
	The type of instruction is determined by first reading the opcode. Then, the Virtual Machine calls the appropriate function by index into an array of function pointers.
    \item Resolve operand \\
	The type of operand must be resolved in this stage. As the goal is to support the typeless language, the source operand gets locally cast to the destination operand type. 
    \item Execute instruction \\
	Once the operands are resolved locally, the Virtual Machine executes the instruction's logic.
    \item Write back \\
	For many instructions, the destination operand (stack index or return value register) gets updated to the result of the execution stage. 
\end{itemize}

\subsection{Flexibility VS Performance}
The simpler instruction set comes with a cost. The Virtual Machine itself is software that needs to translate virtual instruction to real instruction during execution. Therefore, multiple real instructions are generated under the hood to execute one virtual instruction. This was one tradeoff between flexibility and performance.

\section{Assembler}
Although adding the assembler between the compiler and virtual machine was not necessary, the assembler was added for two reasons: (1) to gain more experience in applying compiler theory with a simpler problem domain before implementing the compiler, and (2) to simplify the code generation step for the compiler. The assembler was divided into 3 parts in the following order: lexer, parser, and code generator.

\subsection{Lexer}
The lexer was the starting point of the assembler implementation, where it transforms the input character stream from the text file to the lexeme/token stream needed for the parser. The lexical analysis process was implemented using a state machine, where the following regular expressions define rules for grouping characters into a lexeme.

\subsection{Parser}
The parsing process was designed to complete in 2 passes. In the first pass, the parser record labels, identifiers, and function into corresponding tables. Instructions are ignored in the first pass. In the second pass, the parser evaluates the instruction. When an instruction hold operand of type labels, identifiers, or function, the operand gets replaced with the numeric value (either instruction address for labels/function or stack address for identifiers). 2 pass parsing was needed to address forward referencing.

The parser follows the top-down approach, resembling recursive descent parsing. However, due to the grammar structure of the assembly language, identifying the first token on the line was enough to determine the syntax and semantics of the line. When the first token is directive, the parser follows the grammar rules and recurses to the appropriate function. However, for most cases when the first token is an opcode, syntax and semantic analysis are simple as matching the instruction template of the read opcode from the instruction lookup table.

In the end, the parser output a linked list of instructions, string, and function tables that are used by the code generator to output binary executable.

\subsection{Code generator}
The code generator uses the linked list of instructions, string, and function tables outputted by the parser to generate a binary executable. The format of the binary executable is shown below.

\section{Compiler}

\section{Test strategy}


\section{Test results}


\section{Limitation}


\section{Future work}


%%
%% The acknowledgments section is defined using the "acks" environment
%% (and NOT an unnumbered section). This ensures the proper
%% identification of the section in the article metadata, and the
%% consistent spelling of the heading.
\begin{acks}
To Dr.Gulzar, for supervising the independent study and providing guidance.
\end{acks}

%%
%% The next two lines define the bibliography style to be used, and
%% the bibliography file.
\bibliographystyle{ACM-Reference-Format}
\bibliography{references}

%%
%% If your work has an appendix, this is the place to put it.
\appendix

\section{Instruction set}
\subsection{Memory}
\begin{center}
\begin{tabular}{|c|c|c|c|}
    \hline
    Instruction & Opcode & Operand count & Operands \\
    \hline
    mov & 0 & 2 & DESTINATION, SOURCE \\
    \hline
\end{tabular}
\end{center}

\subsection{Arithmetic}
\begin{center}
\begin{tabular}{|c|c|c|c|}
    \hline
    Instruction & Opcode & Operand count & Operands \\
    \hline
    add & 1 & 2 & DESTINATION, SOURCE \\
    sub & 2 & 2 & DESTINATION, SOURCE \\
    mul & 3 & 2 & DESTINATION, SOURCE \\
    div & 4 & 2 & DESTINATION, SOURCE \\
    mod & 5 & 2 & DESTINATION, SOURCE \\
    exp & 6 & 2 & DESTINATION, SOURCE \\
    neg & 7 & 1 & DESTINATION \\
    inc & 8 & 1 & DESTINATION \\
    dec & 9 & 1 & DESTINATION \\
    \hline
\end{tabular}
\end{center}

\subsection{Bitwise}
\subsection{String Processing}
\subsection{Conditional Branching}
\subsection{Stack Interface}
\subsection{Function Interface}
\subsection{Directives}



\end{document}
\endinput
%%
%% End of file `sample-authordraft.tex'.
